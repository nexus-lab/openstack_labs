\documentclass[letterpaper, 12pt]{article}
% \usepackage[showframe, margin=1in, top=0.25in, bottom=0.25in, includeheadfoot, headheight=0.5in]{geometry}
\usepackage[margin=1in, top=0.25in, bottom=0.25in, includeheadfoot, headheight=0.5in]{geometry}

\AddToHook{cmd/section/before}{\clearpage}

\usepackage[table]{xcolor}
\colorlet{listingback}{gray!20}
\definecolor{headingcolor}{RGB}{110,34,54}

\usepackage{fancyhdr}
\renewcommand{\sectionmark}[1]{\markboth{#1}{#1}}

% Used to detect whether a section is an appendix to print the right thing in the footer
\usepackage{etoolbox}
\newtoggle{inappendix}
\pretocmd{\appendix}{\clearpage\toggletrue{inappendix}}{}{}

% Save standard definitions for head and foot rules (lines separating header and footer from text)
\let\HeadRule\headrule
\let\FootRule\footrule
% Add color to the standard definitions
\renewcommand{\headrule}{\color{headingcolor}\HeadRule}
\renewcommand{\footrule}{\textcolor{headingcolor}{\FootRule}}

% IMPORTANT: This command should not be called directly. Use \preamble.
% Macro to insert the title page for each lab.
% The argument is the title of the lab.
\newcommand{\inserttitlepage}[1]
{
    \begin{titlepage}
    \centering
    \includegraphics[scale=0.5]{images/nexus_lab_logo.png}

    \vspace*{\baselineskip}

    \textbf{\Large OpenStack Labs}

    \vspace*{\baselineskip}

    \textbf{\Large #1}
    \vspace*{\fill}
\end{titlepage}
}

% IMPORTANT: This command should not be called directly. Use \preamble.
% Macro to define header and footer for each lab.
% The argument is the title of the lab.
\newcommand{\headfoot}[1]
{
    \fancypagestyle{fancy}
    {
        \fancyhf{}
        \fancyhead[L]{\footnotesize #1}
        \fancyhead[R]{\includegraphics[height=0.85\headheight]{images/nexus_lab_logo.png}}
        \fancyfoot[L]{%
            \footnotesize%
            \ifnum\value{section}>0%
            \iftoggle{inappendix}{Appendix \thesection: \rightmark}{Section \thesection: \rightmark}%
            \fi}
        \fancyfoot[R]{\footnotesize\thepage}
        \renewcommand{\headrulewidth}{1.5pt}
        \renewcommand{\footrulewidth}{1.5pt}
    }
}

% Macro to insert title page, define header and footer, and insert table of contents and about section for each lab.
% The argument is the title of the lab.
\newcommand{\preamble}[1]
{
    \pagenumbering{roman}
    \inserttitlepage{#1}
    \headfoot{#1}

    % Insert table of contents
    \pagestyle{fancy}
    \tableofcontents
    \clearpage

    \section*{About This Document}
    \label{sec:about_this_document}
    \begin{itemize}
        \item This document was developed by a team at the University of Tennessee at Chattanooga led by Dr. Mengjun Xie
        (\href{mailto:mengjun-xie@utc.edu}{\textbf{mengjun-xie@utc.edu}}).
        \item The development of this document was supported by a National Centers of Academic Excellence in Cybersecurity Grant (\#H98230-20-1-0351), housed at the National Security Agency.
        \item This document is licensed with a Creative Commons Attribution 4.0 International License.
    \end{itemize}
    \clearpage
}

% Macro to insert the Lab Settings page for each lab. Call after the Introduction and Objectives sections.
\newcommand{\labsettings}
{
    \section*{Lab Settings}
    \label{sec:lab_settings}
    \addcontentsline{toc}{section}{\nameref{sec:lab_settings}}
    The information in the table below will be needed in order to complete the lab.
    The task sections below provide details on the use of this information.
    \begin{table*}[htbp]
        \centering
        \begin{tabular}{|c|c|c|c|}
            \hline
            \rowcolor{gray!20} \textbf{Virtual Machine} & \textbf{IP Address} & \textbf{Account} & \textbf{Password} \\
            \hline
            \multirow{2}{*}{\texttt{workstation}} & \multirow[t]{2}{*}{\texttt{ens3: 192.168.1.21}}  & \multirow{2}{*}{\texttt{ubuntu}} & \multirow{2}{*}{\texttt{ubuntu}} \\
                                                  & \multirow[t]{2}{*}{\texttt{ens4: 172.25.250.21}} &                                  &                                  \\
            \hline
            \multirow{2}{*}{\texttt{devstack}}    & \multirow[t]{2}{*}{\texttt{ens3: 192.168.20}}    & \multirow{2}{*}{\texttt{ubuntu}} & \multirow{2}{*}{\texttt{ubuntu}} \\
                                                  & \multirow[t]{2}{*}{\texttt{ens4: 172.25.250.20}} &                                  &                                  \\
            \hline
        \end{tabular}
    \end{table*}
    \clearpage

    % IMPORTANT(lucas): If another frontmatter section ever gets placed after this, this command needs to be moved
    % to the end of that section.
    % I have placed this here and not in each lab purely for convenience and to ensure I don't forget any.
    \pagenumbering{arabic}
}

% Sans-serif font
\renewcommand{\familydefault}{\sfdefault}
\newcommand{\texttildemid}{{\raisebox{0.5ex}{\texttildelow}}}

\usepackage{enumitem}
\renewcommand{\labelenumi}{\textbf{\thesection.\arabic{enumi}.}}

% Try to forbid widows and orphans
\widowpenalty10000
\clubpenalty10000

\usepackage{graphicx}
\usepackage{hyperref}
\hypersetup{colorlinks=true,linkcolor=black,urlcolor={[named] headingcolor}}

\usepackage{sectsty}
\sectionfont{\color{headingcolor}}

% Table of Contents
\usepackage{bookmark}
\usepackage[titles]{tocloft}
\usepackage[title]{appendix}
\renewcommand{\cfttoctitlefont}{\Large\bfseries\color{headingcolor}}
\renewcommand{\cftsecfont}{\normalfont\normalsize}
\renewcommand{\cftsecpagefont}{\normalfont\normalsize}
\renewcommand{\cftdotsep}{0} % Make dots small and close together
\renewcommand{\cftsecleader}{\cftdotfill{\cftdotsep}} % Add dots after section titles
% Make dots go all the way to the page number
\renewcommand{\cftsecfillnum}[1]{{\cftsecleader}\nobreak{\cftsecpagefont #1}\cftsecafterpnum\par}

\usepackage{multirow}
\setlength{\tabcolsep}{16pt}
\renewcommand{\arraystretch}{1.1}

% For nice-looking boxes
\usepackage[most]{tcolorbox}
\usepackage{listings}
\usepackage{lstautogobble}
\lstset{
  frame=none,
  language=Bash,
  showstringspaces=false,
  basicstyle={\linespread{1.1}\footnotesize\ttfamily\selectfont},
  numbers=none,
  breaklines=true,
  breakatwhitespace=true,
  tabsize=3,
  columns=fullflexible,
  keepspaces=true,
  escapeinside={(*@}{@*)},
  literate={~}{{\texttildemid}}{1}
           {\#}{\#}{1},
  autogobble=true
}

\tcolorboxenvironment{lstlisting}
{
    spartan,
    colframe=gray!50,
    boxsep=0mm,
    left=1mm,
    right=1mm,
    top=-1mm,
    bottom=-1mm,
    colback=gray!20
}

% Hacky solution for now, would like to have just one environment and make several tcolorboxes by passing different
% colors as parameters, but that is giving errors
\makeatletter
\tcbset{
  note/.style={%
        enhanced,
        breakable,
        colback=blue!10!white,
        colframe=blue!80!white,
        attach boxed title to top left={yshift*=-\tcboxedtitleheight},
        title={#1},
        boxed title size=title,
        boxed title style={%
            sharp corners,
            rounded corners=northwest,
            colback=tcbcolframe,
            boxrule=0pt,
        },
        underlay boxed title={%
            \path[fill=tcbcolframe] (title.south west)--(title.south east)
                to[out=0, in=180] ([xshift=5mm]title.east)--
                (title.center-|frame.east)
                [rounded corners=\kvtcb@arc] |-
                (frame.north) -| cycle;
        },
    }
}
\makeatother

\makeatletter
\tcbset{
    stop/.style={%
        enhanced,
        breakable,
        colback=white,
        colback=red!10!white,
        colframe=red!80!white,
        attach boxed title to top left={yshift*=-\tcboxedtitleheight},
        title={#1},
        boxed title size=title,
        boxed title style={%
            sharp corners,
            rounded corners=northwest,
            colback=tcbcolframe,
            boxrule=0pt,
        },
        underlay boxed title={%
            \path[fill=tcbcolframe] (title.south west)--(title.south east)
                to[out=0, in=180] ([xshift=5mm]title.east)--
                (title.center-|frame.east)
                [rounded corners=\kvtcb@arc] |-
                (frame.north) -| cycle;
        },
    }
}
\makeatother

\makeatletter
\tcbset{
    tip/.style={%
        enhanced,
        breakable,
        colback=white,
        colback=green!10,
        colframe=green!70!black,
        attach boxed title to top left={yshift*=-\tcboxedtitleheight},
        fonttitle=\bfseries,
        title={#1},
        boxed title size=title,
        boxed title style={%
            sharp corners,
            rounded corners=northwest,
            colback=tcbcolframe,
            boxrule=0pt,
        },
        underlay boxed title={%
            \path[fill=tcbcolframe] (title.south west)--(title.south east)
                to[out=0, in=180] ([xshift=5mm]title.east)--
                (title.center-|frame.east)
                [rounded corners=\kvtcb@arc] |-
                (frame.north) -| cycle;
        },
    }
}
\makeatother

% The commands below define environments for colored boxes. They are used like
% \begin{notebox}
% ...
% \end{notebox}
\newtcolorbox{notebox}{note={Note}}
\newtcolorbox{stopbox}{stop={Stop}}
\newtcolorbox{tipbox}{tip={Tip}}

\begin{document}
\preamble{Lab 01: Launching an Instance}

\section*{Introduction}\label{sec:introduction}
\addcontentsline{toc}{section}{\nameref{sec:introduction}}
OpenStack is a cloud operating system that manages large pools of compute, storage, and networking resources throughout a datacenter.
These resources can be managed through a web-based dashboard called the \textit{Horizon Dashboard}, or through a command-line interface called the \textit{OpenStack Unified CLI}.
This lab series provides a hands-on tour of key OpenStack features and components.
Each lab focuses on a different aspect of OpenStack, and the final lab is a capstone project that integrates the skills and knowledge you have gained.

In this lab, you will launch an instance with the \textit{Horizon Dashboard} and the \textit{OpenStack Unified CLI}.

\section*{Objectives}\label{sec:objectives}
\addcontentsline{toc}{section}{\nameref{sec:objectives}}
\begin{itemize}[itemsep=0pt]
    \item Use the \textit{Horizon Dashboard}.
    \item Launch an instance with the \textit{Horizon Dashboard}.
    \item Use the \textit{OpenStack Unified CLI}.
    \item Launch an instance with the \textit{OpenStack Unified CLI}.
\end{itemize}
\clearpage

\labsettings

%%%%%%%%%%%
% Section 1
%%%%%%%%%%%
\section{Launching an Instance with the Horizon Dashboard}\label{launching-an-instance-with-the-horizon-dashboard}
In this task, you will launch an instance with the \textit{Horizon Dashboard}.

\begin{enumerate}
    \begin{labstep}
        Log into the \textbf{workstation} machine as the \textbf{ubuntu} user with password \textbf{ubuntu}.

        \begin{center}
            \includegraphics[width=\linewidth]{images/part1/step1.png}
        \end{center}
    \end{labstep}

    \begin{labstep}
        Launch the graphical user interface.

        \begin{lstlisting}
            ubuntu@workstation:~$ startx
        \end{lstlisting}

        \begin{center}
            \includegraphics[width=\linewidth]{images/part1/step2.png}
        \end{center}
    \end{labstep}

    \begin{labstep}
        Open the web browser.

        \begin{center}
            \includegraphics[scale=0.75]{images/part1/step3.png}
        \end{center}
    \end{labstep}

    \begin{labstep}
        Enter the IP address of the \textbf{devstack} machine (\textbf{192.168.1.20}) into the address bar, and log into the OpenStack Horizon Dashboard with username \textbf{admin} and password \textbf{secret}.

        \begin{center}
            \includegraphics[scale=0.65]{images/part1/step4.png}
        \end{center}
    \end{labstep}

    \begin{labstep}
        Click the dropdown menu in the top left corner of the webpage, then select \textbf{demo} as the project.
        Navigate to \textbf{Project $>$ Compute $>$ Instances}, then click \textbf{Launch Instance} in the top right corner.

        \begin{center}
            \includegraphics[width=\linewidth]{images/part1/step5.png}
        \end{center}
    \end{labstep}

    \begin{labstep}
        In the \textit{Instance Name} field, type \textbf{prod-instance}, and leave the other fields with their default values.
        Click \textbf{Next}.

        \begin{center}
            \includegraphics[width=\linewidth]{images/part1/step6.png}
        \end{center}
    \end{labstep}

    \begin{labstep}
        In the \textit{Select Boot Source} dropdown, \textbf{Image} should already be selected.
        Set \textit{Create New Volume} to \textbf{No}, and scroll down (if needed) to click the $\uparrow$ icon beside of \textbf{ubuntu} to use \textbf{ubuntu} as the image.
        Click \textbf{Next}.

        \begin{center}
            \includegraphics[width=\linewidth]{images/part1/step7.png}
        \end{center}
    \end{labstep}

    \begin{stopbox}
        Before proceeding to the next step, confirm that \textbf{ubuntu} appears underneath the \textit{Allocated} section.
    \end{stopbox}

    \begin{labstep}
        Scroll down (if needed) and click the $\uparrow$ icon beside the \textbf{m1.small} flavor.
        Click \textbf{Next}.

        \begin{center}
            \includegraphics[width=\linewidth]{images/part1/step8.png}
        \end{center}
    \end{labstep}

    \begin{stopbox}
        Before proceeding to the next step, confirm that \textbf{m1.small} appears underneath the \textit{Allocated} section.
    \end{stopbox}

    \begin{labstep}
        Click the $\uparrow$ icon beside the \textbf{shared} network.

        \begin{center}
            \includegraphics[width=\linewidth]{images/part1/step9.png}
        \end{center}
    \end{labstep}

    \begin{stopbox}
        Before proceeding to the next step, confirm that \textbf{shared} appears underneath the \textit{Allocated} section.
    \end{stopbox}

    \begin{labstep}
        If all required fields have been set, the \textbf{Launch Instance} button in the bottom right corner should now be available.
        Click \textbf{Launch Instance}.

        \begin{center}
            \includegraphics[width=\linewidth]{images/part1/step10.png}
        \end{center}
    \end{labstep}

    \begin{stopbox}
        Wait for the \textit{Power State} of \textbf{prod-instance} to display the status of \textit{Running} before continuing to the next step.
    \end{stopbox}

    \begin{labstep}
        To open the console of \textbf{prod-instance} in a new tab, right-click the name \textbf{prod-instance} and select \textbf{Open Link in New Tab}, or middle-click (press in the mouse wheel) the name \textbf{prod-instance}.

        \begin{center}
            \includegraphics[width=\linewidth]{images/part1/step11.png}
        \end{center}
    \end{labstep}

    \begin{labstep}
        In the new tab, click the \textit{Console} tab.
        Optionally, to make the console take up the whole tab, click the \textbf{Click here to show only console} link.

        \begin{center}
            \includegraphics[width=\linewidth]{images/part1/step12.png}
        \end{center}
    \end{labstep}

    \begin{labstep}
        Log into the console as \textbf{root} with the password \textbf{secret}.
    \end{labstep}

    \begin{notebox}
        It may take several minutes for the instance to fully boot up and present a login prompt.
    \end{notebox}

    \begin{labstep}
        In the console, ping \textbf{192.168.233.2} (DHCP server) to verify connectivity.
        \begin{lstlisting}
            $ ping -c3 192.168.233.2
        \end{lstlisting}

        \begin{center}
            \includegraphics[width=\linewidth]{images/part1/step14.png}
        \end{center}
    \end{labstep}

    \begin{notebox}
        You should receive three successful ping replies.
    \end{notebox}

    \begin{labstep}
        Close the console tab for \textbf{prod-instance}.
    \end{labstep}

    \begin{labstep}
        Focus back on the tab showing instances and delete \textbf{prod-instance}.
        Select the checkbox for \textbf{prod-instance} and click the \textbf{Delete Instances} button.

        \begin{center}
            \includegraphics[width=\linewidth]{images/part1/step16.png}
        \end{center}
    \end{labstep}

    \begin{labstep}
        Confirm the deletion by clicking the \textbf{Delete Instances} button.

        \begin{center}
            \includegraphics[width=\linewidth]{images/part1/step17.png}
        \end{center}
    \end{labstep}

    \begin{labstep}
        Close the web browser.
        Continue to the next task.
    \end{labstep}
\end{enumerate}

%%%%%%%%%%%
% Section 2
%%%%%%%%%%%
\section{Running the OpenStack Unified CLI}\label{sec:running-the-openstack-unified-cli}
In this task, you will launch an instance with the \textit{OpenStack Unified command-line interface (CLI)}.

\begin{enumerate}
    \begin{labstep}
        Open a terminal by clicking the terminal icon in the icon bar at the bottom of the screen.
        A terminal can also be opened by right-clicking the desktop and selecting \textbf{Open Terminal Here}, or by selecting \textbf{Applications} at the top left of the screen, then selecting \textbf{Terminal Emulator}.

        \begin{center}
            \includegraphics[width=\linewidth]{images/part2/step1.png}
        \end{center}
    \end{labstep}

    \begin{labstep}
        Source the \textbf{keystonerc-admin} file.
        This will provide a connection with the local OpenStack service with the credentials of the \textbf{admin} user, and the command prompt will change to indicate the current OpenStack user whose credentials are keyed in.
        \begin{lstlisting}
            ubuntu@workstation:~$ source ~/keystonerc-admin
        \end{lstlisting}

        \begin{center}
            \includegraphics[width=\linewidth]{images/part2/step2.png}
        \end{center}
    \end{labstep}

    \begin{notebox}
        \textbf{keystonerc} files will be discussed in more depth in a future lab.
    \end{notebox}

    \begin{labstep}
        The OpenStack CLI is now ready to use, allowing us to create an instance.
        Just like in the Horizon Dashboard, an image, a flavor, and a network are required.
        Before launching an instance, we will list the available options for these resources.
        First, list all available images.
        \begin{lstlisting}
            [ubuntu@workstation (keystone-admin)]:~$ openstack image list
        \end{lstlisting}

        \begin{center}
            \includegraphics[width=\linewidth]{images/part2/step3.png}
        \end{center}
    \end{labstep}

    \begin{labstep}
        List all available flavors.
        \begin{lstlisting}
            [ubuntu@workstation (keystone-admin)]:~$ openstack flavor list
        \end{lstlisting}

        \begin{center}
            \includegraphics[width=\linewidth]{images/part2/step4.png}
        \end{center}
    \end{labstep}

    \begin{labstep}
        Display the details specifically for the \textbf{m1.small} flavor.
        \begin{lstlisting}
            [ubuntu@workstation (keystone-admin)]:~$ openstack flavor show m1.small
        \end{lstlisting}

        \begin{center}
            \includegraphics[width=\linewidth]{images/part2/step5.png}
        \end{center}
    \end{labstep}

    \begin{labstep}
        List all available networks.
        \begin{lstlisting}
            [ubuntu@workstation (keystone-admin)]:~$ openstack network list
        \end{lstlisting}

        \begin{center}
            \includegraphics[width=\linewidth]{images/part2/step6.png}
        \end{center}
    \end{labstep}

    \begin{labstep}
        Create a new instance with the name \textbf{prod-instance}, using \textbf{ubuntu} as the image, \textbf{m1.small} as the flavor, and \textbf{shared} as the network.
        \begin{lstlisting}
            [ubuntu@workstation (keystone-admin)]:~$ openstack server create \
            > --image ubuntu \
            > --flavor m1.small \
            > --network shared \
            > prod-instance
        \end{lstlisting}

        \begin{center}
            \includegraphics[width=\linewidth]{images/part2/step7.png}
        \end{center}
    \end{labstep}

    \begin{tipbox}
        When typing the command, make sure there is a space between the last word of the line and \textbf{\textbackslash}, and press \textbf{Enter} to get the \textbf{$>$} and continue typing the rest of the command.
    \end{tipbox}

    \begin{labstep}
        Use the \textbf{openstack server list} command to list all the available instances.
        \begin{lstlisting}
            [ubuntu@workstation (keystone-admin)]:~$ openstack server list
        \end{lstlisting}

        \begin{center}
            \includegraphics[width=\linewidth]{images/part2/step8.png}
        \end{center}
    \end{labstep}

    \begin{notebox}
        The UUID in the \textit{ID} field and the IP address in the \textit{Networks} field may differ from the screenshot provided.
    \end{notebox}

    \begin{labstep}
        Display more details about the instance \textbf{prod-instance}.
        \begin{lstlisting}
            [ubuntu@workstation (keystone-admin)]:~$ openstack server show prod-instance
        \end{lstlisting}

        \begin{center}
            \includegraphics[width=\linewidth]{images/part2/step9.png}
        \end{center}
    \end{labstep}

    \begin{tipbox}
        The UUID for the instance \textbf{prod-instance} can be used in place of \textbf{prod-instance} in the above command to identify the instance.
    \end{tipbox}

    \begin{labstep}
        Display the instance's console URL.
        Then, right-click the URL and select \textbf{Open Link}.
        \begin{lstlisting}
            [ubuntu@workstation (keystone-admin)]:~$ openstack console url show prod-instance
        \end{lstlisting}

        \begin{center}
            \includegraphics[width=\linewidth]{images/part2/step10.png}
        \end{center}
    \end{labstep}

    \begin{labstep}
        The web browser will open directly to the instance's console through noVNC.
        Log into \textbf{prod-instance} using \textbf{root} as the username and \textbf{secret} as the password.
        Then use the \textbf{ping} command to verify connectivity with the DHCP server (\textbf{192.168.233.2}).
        \begin{lstlisting}
            $ ping -c3 192.168.233.2
        \end{lstlisting}

        \begin{center}
            \includegraphics[width=\linewidth]{images/part2/step11.png}
        \end{center}
    \end{labstep}

    \begin{notebox}
        You should receive three successful ping replies.
    \end{notebox}

    \begin{labstep}
        Close the web browser and change focus back to the previous terminal window.
    \end{labstep}

    \begin{labstep}
        The instance is now ready to be deleted, but first list the servers so that the effect of the next step can be observed.
        \begin{lstlisting}
            [ubuntu@workstation (keystone-admin)]:~$ openstack server list
        \end{lstlisting}

        \begin{center}
            \includegraphics[width=\linewidth]{images/part2/step13.png}
        \end{center}
    \end{labstep}

    \begin{labstep}
        Delete the instance.
        \begin{lstlisting}
            [ubuntu@workstation (keystone-admin)]:~$ openstack server delete prod-instance
        \end{lstlisting}

        \begin{center}
            \includegraphics[width=\linewidth]{images/part2/step14.png}
        \end{center}
    \end{labstep}

    \begin{labstep}
        Ensure that the instance was deleted by seeing that the server list is empty.
        \begin{lstlisting}
            [ubuntu@workstation (keystone-admin)]:~$ openstack server list
        \end{lstlisting}

        \begin{center}
            \includegraphics[width=\linewidth]{images/part2/step15.png}
        \end{center}
    \end{labstep}

    \begin{labstep}
        The lab is now complete.
    \end{labstep}

\end{enumerate}

%%%%%%%%%$
% Appendix
%%%%%%%%%$
\appendix
\section{OpenStack Unified CLI Help}

The OpenStack Unified CLI has many commands.
Thankfully, they follow a predictable form:
\begin{lstlisting}
    openstack [<global-options>] <object-1> <action> [<object-2>] [<command-arguments>]
\end{lstlisting}
\noindent
Items in square brackets indicate that they are not present in every command.
For instance,
\begin{lstlisting}
    openstack image list
\end{lstlisting}
\noindent
contains only one object and an action.
We will encounter more complex commands in the future, but it is good to keep this common structure in mind.
For more information on the structure of OpenStack CLI commands, visit the \href{https://docs.openstack.org/python-openstackclient/latest/cli/commands.html}{\textbf{Command Structure Documentation}}.

To see what actions can be performed on a given object, it is always possible to run a command structured like this:
\begin{lstlisting}
    openstack <object> --help
\end{lstlisting}
\noindent
For instance, to see what actions you can perform on instances, you can run the command
\begin{lstlisting}
    openstack instance --help
\end{lstlisting}
\noindent
Other objects include \textbf{image}, \textbf{flavor}, \textbf{network}, and many more that will appear throughout the labs.

Additionally, to see what parameters you can pass to an action, you can run a command structured like this:
\begin{lstlisting}
    openstack <object> <action> --help
\end{lstlisting}
\noindent
for instance, to see what parameters you can set on an instance, you can run the command
\begin{lstlisting}
    openstack server set --help
\end{lstlisting}
\noindent
Other object actions include \textbf{server list}, \textbf{flavor show}, \textbf{network create}, and many more that will appear throughout the labs.

\end{document}
