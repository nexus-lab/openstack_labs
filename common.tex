% \usepackage[showframe, margin=1in, top=0.25in, bottom=0.25in, includeheadfoot, headheight=0.5in]{geometry}
\usepackage[margin=1in, top=0.25in, bottom=0.25in, includeheadfoot, headheight=0.5in]{geometry}
\usepackage[title]{appendix}
\usepackage{bookmark}
\usepackage{enumitem}
\usepackage{etoolbox}
\usepackage{fancyhdr}
\usepackage{graphicx}
\usepackage{hyperref}
\usepackage{lstautogobble}
\usepackage{listings}
\usepackage{multirow}
\usepackage{sectsty}
\usepackage{siunitx}
\usepackage[most]{tcolorbox}
\usepackage[titles]{tocloft}
\usepackage[table]{xcolor}

% --- Colors ---
\colorlet{listingback}{gray!20}
\definecolor{headingcolor}{RGB}{110,34,54}

% ===================================================
%              Package Configuration
% ===================================================
\hypersetup{colorlinks=true,linkcolor=black,urlcolor={[named] headingcolor}}

\sisetup{detect-all, mode=text} % Match current font instead of using math mode.
\DeclareSIUnit{\vcpu}{VCPU}
\DeclareSIUnit{\vcpus}{VCPUs}

\renewcommand{\sectionmark}[1]{\markboth{#1}{#1}}
\AddToHook{cmd/section/before}{\clearpage}

% Save standard definitions for head and foot rules (lines separating header and footer from text)
\let\HeadRule\headrule
\let\FootRule\footrule
% Add color to the standard definitions
\renewcommand{\headrule}{\color{headingcolor}\HeadRule}
\renewcommand{\footrule}{\textcolor{headingcolor}{\FootRule}}

% --- Table of Contents ---
\renewcommand{\cfttoctitlefont}{\Large\bfseries\color{headingcolor}}
\renewcommand{\cftsecfont}{\normalfont\normalsize}
\renewcommand{\cftsecpagefont}{\normalfont\normalsize}
\renewcommand{\cftdotsep}{0} % Make dots small and close together
\renewcommand{\cftsecleader}{\cftdotfill{\cftdotsep}} % Add dots after section titles
\renewcommand{\cftsecfillnum}[1]{{\cftsecleader}\nobreak{\cftsecpagefont#1}\cftsecafterpnum\par} % Make dots go all the way to the page number

% --- Listings ---
\lstset{
  frame=none,
  language=Bash,
  showstringspaces=false,
  basicstyle={\linespread{1.1}\footnotesize\ttfamily\selectfont},
  numbers=none,
  breaklines=true,
  breakatwhitespace=true,
  tabsize=3,
  columns=fullflexible,
  keepspaces=true,
  literate={~}{{\texttildemid}}{1}
           {\#}{\#}{1},
  autogobble=true
}
\tcolorboxenvironment{lstlisting}
{
    spartan,
    colframe=gray!50,
    boxsep=0mm,
    left=1mm,
    right=1mm,
    top=-1mm,
    bottom=-1mm,
    colback=gray!20
}

% --- Layout Settings ---
% Sans-serif font
\renewcommand{\familydefault}{\sfdefault}
\newcommand{\texttildemid}{{\raisebox{0.5ex}{\texttildelow}}}

\renewcommand{\labelenumi}{\textbf{\thesection.\arabic{enumi}.}}

% Section heading color (sectsty)
\sectionfont{\color{headingcolor}}

% Try to forbid widows and orphans
\widowpenalty10000
\clubpenalty10000

% Table spacing
\setlength{\tabcolsep}{16pt}
\renewcommand{\arraystretch}{1.1}

% Detect whether a section is an appendix to print the right thing in the footer
\newtoggle{inappendix}
\pretocmd{\appendix}{\clearpage\toggletrue{inappendix}}{}{}

% ===================================================
%               Document Macros
% ===================================================

% IMPORTANT: This command should not be called directly. Use \preamble.
% Macro to insert the title page for each lab.
% The argument is the title of the lab.
\newcommand{\inserttitlepage}[1]
{
    \begin{titlepage}
    \centering
    \includegraphics[scale=0.5]{images/nexus_lab_logo.png}

    \vspace*{\baselineskip}

    \textbf{\Large OpenStack Labs}

    \vspace*{\baselineskip}

    \textbf{\Large #1}
    \vspace*{\fill}
    \end{titlepage}
}

% IMPORTANT: This command should not be called directly. Use \preamble.
% Macro to define header and footer for each lab.
% The argument is the title of the lab.
\newcommand{\headfoot}[1]
{
    \fancypagestyle{fancy}
    {
        \fancyhf{}
        \fancyhead[L]{\footnotesize #1}
        \fancyhead[R]{\includegraphics[height=0.85\headheight]{images/nexus_lab_logo.png}}
        \fancyfoot[L]{%
            \footnotesize%
            \ifnum\value{section}>0%
            \iftoggle{inappendix}{Appendix \thesection: \rightmark}{Section \thesection: \rightmark}%
            \fi}
        \fancyfoot[R]{\footnotesize\thepage}
        \renewcommand{\headrulewidth}{1.5pt}
        \renewcommand{\footrulewidth}{1.5pt}
    }
}

% Macro to insert title page, define header and footer, and insert table of contents and about section for each lab.
% The argument is the title of the lab.
\newcommand{\preamble}[1]
{
    \pagenumbering{roman}
    \inserttitlepage{#1}
    \headfoot{#1}

    % Insert table of contents
    \pagestyle{fancy}
    \tableofcontents
    \clearpage

    \section*{About This Document}\label{sec:about-this-document}
    \begin{itemize}
        \item This document was developed by a team at the University of Tennessee at Chattanooga led by Dr.\ Mengjun Xie
        (\href{mailto:mengjun-xie@utc.edu}{\textbf{mengjun-xie@utc.edu}}).
        \item The development of this document was supported by a National Centers of Academic Excellence in Cybersecurity Grant (\#H98230-20-1-0351), housed at the National Security Agency. % chktex 8
        \item This document is licensed with a Creative Commons Attribution 4.0 International License.
    \end{itemize}
    \clearpage
}

% Macro to insert the Lab Settings page for each lab. Call after the Introduction and Objectives sections.
\newcommand{\labsettings}
{
    \section*{Lab Settings}\label{sec:lab-settings}
    \addcontentsline{toc}{section}{\nameref{sec:lab-settings}}
    The information in the table below will be needed in order to complete the lab.
    The task sections below provide details on the use of this information.
    \begin{table*}[htbp]
        \centering
        \begin{tabular}{|c|c|c|c|} % chktex 44
            \hline % chktex 44
            \rowcolor{gray!20} \textbf{Virtual Machine} & \textbf{IP Address} & \textbf{Account} & \textbf{Password} \\
            \hline % chktex 44
            \multirow{2}{*}{\texttt{workstation}} & \multirow[t]{2}{*}{\texttt{ens3: 192.168.1.21}}  & \multirow{2}{*}{\texttt{ubuntu}} & \multirow{2}{*}{\texttt{ubuntu}} \\
                                                  & \multirow[t]{2}{*}{\texttt{ens4: 172.25.250.21}} &                                  &                                  \\
            \hline % chktex 44
            \multirow{2}{*}{\texttt{devstack}}    & \multirow[t]{2}{*}{\texttt{ens3: 192.168.20}}    & \multirow{2}{*}{\texttt{ubuntu}} & \multirow{2}{*}{\texttt{ubuntu}} \\
                                                  & \multirow[t]{2}{*}{\texttt{ens4: 172.25.250.20}} &                                  &                                  \\
            \hline % chktex 44
        \end{tabular}
    \end{table*}
    \clearpage

    % IMPORTANT(lucas): If another frontmatter section ever gets placed after this, this command needs to be moved
    % to the end of that section.
    % I have placed this here and not in each lab purely for convenience and to ensure I don't forget any.
    \pagenumbering{arabic}
}

% ===================================================
%          Custom Box Environment Definitions
% ===================================================

\makeatletter
\newcommand{\definebox}[4]{% #1 = box environment name, #2 = frame color, #3 = background color, #4 box title
    \newtcolorbox{#1}{%
        enhanced,
        breakable=false,
        colframe={#2},
        colback={#3},
        attach boxed title to top left={yshift*=-\tcboxedtitleheight},
        title={#4},
        boxed title size=title,
        boxed title style={%
            sharp corners,
            rounded corners=northwest,
            colback=tcbcolframe,
            boxrule=0pt,
        },
        underlay boxed title={%
            \path[fill=tcbcolframe] (title.south west)--(title.south east) % chktex 36
                to[out=0, in=180] ([xshift=5mm]title.east)--
                (title.center-|frame.east)
                [rounded corners=\kvtcb@arc] |-
                (frame.north) -| cycle;
        },
    }
}
\makeatother

% The commands below define environments for colored boxes. They are used like
% \begin{notebox}
% ...
% \end{notebox}
\definebox{notebox}{blue!80!white}{blue!10!white}{Note}
\definebox{tipbox}{green!70!black}{green!10}{Tip}
\definebox{stopbox}{red!80!white}{red!10!white}{Stop}

% ===================================================
%               Lab Step Environment
% ===================================================

% NOTE(lucas): This environment should wrap every instruction in the lab, including any listing or figure that follows.
% Anything inside this environment will be kept on the same page.
% Because it uses \item, this environment must be inside an enumerate or itemize environment.
% This must be an environment and not a command because the lstlisting environment cannot be used inside of a command.
% Also, do not put elements like noteboxes inside this environment because they can be messed up in some cases.
\NewDocumentEnvironment{labstep}{}
{%
    \begin{minipage}{\linewidth}
    \item
}
{%
    \end{minipage}
}
