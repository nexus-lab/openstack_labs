\documentclass[letterpaper, 12pt]{article}
% \usepackage[showframe, margin=1in, top=0.25in, bottom=0.25in, includeheadfoot, headheight=0.5in]{geometry}
\usepackage[margin=1in, top=0.25in, bottom=0.25in, includeheadfoot, headheight=0.5in]{geometry}

\AddToHook{cmd/section/before}{\clearpage}

\usepackage[table]{xcolor}
\colorlet{listingback}{gray!20}
\definecolor{headingcolor}{RGB}{110,34,54}

\usepackage{fancyhdr}
\renewcommand{\sectionmark}[1]{\markboth{#1}{#1}}

% Used to detect whether a section is an appendix to print the right thing in the footer
\usepackage{etoolbox}
\newtoggle{inappendix}
\pretocmd{\appendix}{\clearpage\toggletrue{inappendix}}{}{}

% Save standard definitions for head and foot rules (lines separating header and footer from text)
\let\HeadRule\headrule
\let\FootRule\footrule
% Add color to the standard definitions
\renewcommand{\headrule}{\color{headingcolor}\HeadRule}
\renewcommand{\footrule}{\textcolor{headingcolor}{\FootRule}}

% IMPORTANT: This command should not be called directly. Use \preamble.
% Macro to insert the title page for each lab.
% The argument is the title of the lab.
\newcommand{\inserttitlepage}[1]
{
    \begin{titlepage}
    \centering
    \includegraphics[scale=0.5]{images/nexus_lab_logo.png}

    \vspace*{\baselineskip}

    \textbf{\Large OpenStack Labs}

    \vspace*{\baselineskip}

    \textbf{\Large #1}
    \vspace*{\fill}
\end{titlepage}
}

% IMPORTANT: This command should not be called directly. Use \preamble.
% Macro to define header and footer for each lab.
% The argument is the title of the lab.
\newcommand{\headfoot}[1]
{
    \fancypagestyle{fancy}
    {
        \fancyhf{}
        \fancyhead[L]{\footnotesize #1}
        \fancyhead[R]{\includegraphics[height=0.85\headheight]{images/nexus_lab_logo.png}}
        \fancyfoot[L]{%
            \footnotesize%
            \ifnum\value{section}>0%
            \iftoggle{inappendix}{Appendix \thesection: \rightmark}{Section \thesection: \rightmark}%
            \fi}
        \fancyfoot[R]{\footnotesize\thepage}
        \renewcommand{\headrulewidth}{1.5pt}
        \renewcommand{\footrulewidth}{1.5pt}
    }
}

% Macro to insert title page, define header and footer, and insert table of contents and about section for each lab.
% The argument is the title of the lab.
\newcommand{\preamble}[1]
{
    \pagenumbering{roman}
    \inserttitlepage{#1}
    \headfoot{#1}

    % Insert table of contents
    \pagestyle{fancy}
    \tableofcontents
    \clearpage

    \section*{About This Document}
    \label{sec:about_this_document}
    \begin{itemize}
        \item This document was developed by a team at the University of Tennessee at Chattanooga led by Dr. Mengjun Xie
        (\href{mailto:mengjun-xie@utc.edu}{\textbf{mengjun-xie@utc.edu}}).
        \item The development of this document was supported by a National Centers of Academic Excellence in Cybersecurity Grant (\#H98230-20-1-0351), housed at the National Security Agency.
        \item This document is licensed with a Creative Commons Attribution 4.0 International License.
    \end{itemize}
    \clearpage
}

% Macro to insert the Lab Settings page for each lab. Call after the Introduction and Objectives sections.
\newcommand{\labsettings}
{
    \section*{Lab Settings}
    \label{sec:lab_settings}
    \addcontentsline{toc}{section}{\nameref{sec:lab_settings}}
    The information in the table below will be needed in order to complete the lab.
    The task sections below provide details on the use of this information.
    \begin{table*}[htbp]
        \centering
        \begin{tabular}{|c|c|c|c|}
            \hline
            \rowcolor{gray!20} \textbf{Virtual Machine} & \textbf{IP Address} & \textbf{Account} & \textbf{Password} \\
            \hline
            \multirow{2}{*}{\texttt{workstation}} & \multirow[t]{2}{*}{\texttt{ens3: 192.168.1.21}}  & \multirow{2}{*}{\texttt{ubuntu}} & \multirow{2}{*}{\texttt{ubuntu}} \\
                                                  & \multirow[t]{2}{*}{\texttt{ens4: 172.25.250.21}} &                                  &                                  \\
            \hline
            \multirow{2}{*}{\texttt{devstack}}    & \multirow[t]{2}{*}{\texttt{ens3: 192.168.20}}    & \multirow{2}{*}{\texttt{ubuntu}} & \multirow{2}{*}{\texttt{ubuntu}} \\
                                                  & \multirow[t]{2}{*}{\texttt{ens4: 172.25.250.20}} &                                  &                                  \\
            \hline
        \end{tabular}
    \end{table*}
    \clearpage

    % IMPORTANT(lucas): If another frontmatter section ever gets placed after this, this command needs to be moved
    % to the end of that section.
    % I have placed this here and not in each lab purely for convenience and to ensure I don't forget any.
    \pagenumbering{arabic}
}

% Sans-serif font
\renewcommand{\familydefault}{\sfdefault}
\newcommand{\texttildemid}{{\raisebox{0.5ex}{\texttildelow}}}

\usepackage{enumitem}
\renewcommand{\labelenumi}{\textbf{\thesection.\arabic{enumi}.}}

% Try to forbid widows and orphans
\widowpenalty10000
\clubpenalty10000

\usepackage{graphicx}
\usepackage{hyperref}
\hypersetup{colorlinks=true,linkcolor=black,urlcolor={[named] headingcolor}}

\usepackage{sectsty}
\sectionfont{\color{headingcolor}}

% Table of Contents
\usepackage{bookmark}
\usepackage[titles]{tocloft}
\usepackage[title]{appendix}
\renewcommand{\cfttoctitlefont}{\Large\bfseries\color{headingcolor}}
\renewcommand{\cftsecfont}{\normalfont\normalsize}
\renewcommand{\cftsecpagefont}{\normalfont\normalsize}
\renewcommand{\cftdotsep}{0} % Make dots small and close together
\renewcommand{\cftsecleader}{\cftdotfill{\cftdotsep}} % Add dots after section titles
% Make dots go all the way to the page number
\renewcommand{\cftsecfillnum}[1]{{\cftsecleader}\nobreak{\cftsecpagefont #1}\cftsecafterpnum\par}

\usepackage{multirow}
\setlength{\tabcolsep}{16pt}
\renewcommand{\arraystretch}{1.1}

% For nice-looking boxes
\usepackage[most]{tcolorbox}
\usepackage{listings}
\usepackage{lstautogobble}
\lstset{
  frame=none,
  language=Bash,
  showstringspaces=false,
  basicstyle={\linespread{1.1}\footnotesize\ttfamily\selectfont},
  numbers=none,
  breaklines=true,
  breakatwhitespace=true,
  tabsize=3,
  columns=fullflexible,
  keepspaces=true,
  escapeinside={(*@}{@*)},
  literate={~}{{\texttildemid}}{1}
           {\#}{\#}{1},
  autogobble=true
}

\tcolorboxenvironment{lstlisting}
{
    spartan,
    colframe=gray!50,
    boxsep=0mm,
    left=1mm,
    right=1mm,
    top=-1mm,
    bottom=-1mm,
    colback=gray!20
}

% Hacky solution for now, would like to have just one environment and make several tcolorboxes by passing different
% colors as parameters, but that is giving errors
\makeatletter
\tcbset{
  note/.style={%
        enhanced,
        breakable,
        colback=blue!10!white,
        colframe=blue!80!white,
        attach boxed title to top left={yshift*=-\tcboxedtitleheight},
        title={#1},
        boxed title size=title,
        boxed title style={%
            sharp corners,
            rounded corners=northwest,
            colback=tcbcolframe,
            boxrule=0pt,
        },
        underlay boxed title={%
            \path[fill=tcbcolframe] (title.south west)--(title.south east)
                to[out=0, in=180] ([xshift=5mm]title.east)--
                (title.center-|frame.east)
                [rounded corners=\kvtcb@arc] |-
                (frame.north) -| cycle;
        },
    }
}
\makeatother

\makeatletter
\tcbset{
    stop/.style={%
        enhanced,
        breakable,
        colback=white,
        colback=red!10!white,
        colframe=red!80!white,
        attach boxed title to top left={yshift*=-\tcboxedtitleheight},
        title={#1},
        boxed title size=title,
        boxed title style={%
            sharp corners,
            rounded corners=northwest,
            colback=tcbcolframe,
            boxrule=0pt,
        },
        underlay boxed title={%
            \path[fill=tcbcolframe] (title.south west)--(title.south east)
                to[out=0, in=180] ([xshift=5mm]title.east)--
                (title.center-|frame.east)
                [rounded corners=\kvtcb@arc] |-
                (frame.north) -| cycle;
        },
    }
}
\makeatother

\makeatletter
\tcbset{
    tip/.style={%
        enhanced,
        breakable,
        colback=white,
        colback=green!10,
        colframe=green!70!black,
        attach boxed title to top left={yshift*=-\tcboxedtitleheight},
        fonttitle=\bfseries,
        title={#1},
        boxed title size=title,
        boxed title style={%
            sharp corners,
            rounded corners=northwest,
            colback=tcbcolframe,
            boxrule=0pt,
        },
        underlay boxed title={%
            \path[fill=tcbcolframe] (title.south west)--(title.south east)
                to[out=0, in=180] ([xshift=5mm]title.east)--
                (title.center-|frame.east)
                [rounded corners=\kvtcb@arc] |-
                (frame.north) -| cycle;
        },
    }
}
\makeatother

% The commands below define environments for colored boxes. They are used like
% \begin{notebox}
% ...
% \end{notebox}
\newtcolorbox{notebox}{note={Note}}
\newtcolorbox{stopbox}{stop={Stop}}
\newtcolorbox{tipbox}{tip={Tip}}

\begin{document}
\preamble{Lab 06: Managing an OpenStack Instance}

\section*{Introduction}
\label{sec:introduction}
\addcontentsline{toc}{section}{\nameref{sec:introduction}}
Up to this point, whenever you have launched an instance, its resources and running state have remained mostly constant.
However, OpenStack instances are quite flexible, even after being launched. In this lab, you will launch an instance and
perform several management operations while it is running.

\section*{Objectives}
\label{sec:objectives}
\addcontentsline{toc}{section}{\nameref{sec:objectives}}
\begin{itemize}[itemsep=0pt]
    \item Create a snapshot of an instance.
    \item Manage the running and power state of an instance.
\end{itemize}
\clearpage

\labsettings

%%%%%%%%%%%
% Section 1
%%%%%%%%%%%
\section{Creating a Snapshot Using the Horizon Dashboard}
\label{sec:creating_a_snapshot_web}
In this task, you will launch an instance, modify its configuration, and make a snapshot of it for later use. Taking a
snapshot of an image captures its state and data on disk. An OpenStack snapshot can be used as an image to launch other
instances. There are a couple situations where snapshots might be a useful tool. One is to aid in the backup of instance
or as a precaution before a major change so that the image is easily recoverable. Another use of snapshots is to build
off an existing image to create a better template for new instances. For instance, if you want to launch multiple FTP
servers, instead of launching each one from a basic Ubuntu image and modifying their configurations individually, you
might set up one instance completely, take a snapshot, and use the snapshot as a template for the other instances to
save time and prevent mistakes.

\begin{enumerate}
    \item Log into the \textbf{workstation} machine as the \textbf{ubuntu} user with password \textbf{ubuntu}.

    \begin{center}
        \includegraphics[width=\linewidth]{images/part1/step1.png}
    \end{center}

    \item Launch the graphical user interface.
    \begin{lstlisting}
        ubuntu@workstation:~$ startx
    \end{lstlisting}

    \begin{center}
        \includegraphics[width=\linewidth]{images/part1/step2.png}
    \end{center}

    \item Open the web browser and navigate to \textbf{192.168.1.20}. Log into the dashboard as \textbf{admin} with the
    password \textbf{secret}.

    \item Ensure the \textbf{demo} project is selected. Navigate to \textbf{Project $>$ Compute $>$ Instances}, and
    click \textbf{Launch Instance}.

    \begin{center}
        \includegraphics[width=\linewidth]{images/part1/step4.png}
    \end{center}

    \item In the \textit{Details} tab, enter \textbf{instance1} in the \textit{Instance Name} field and click
    \textbf{Next}.

    \begin{center}
        \includegraphics[width=\linewidth]{images/part1/step5.png}
    \end{center}

    \item In the \textit{Source} tab, make sure \textbf{Image} is selected in the \textit{Select Boot Source} dropdown
    and select \textbf{No} under \textit{Create New Volume}. Select the \textbf{ubuntu} image by clicking the $\uparrow$
    symbol in the same row. Click \textbf{Next}.

    \begin{center}
        \includegraphics[width=\linewidth]{images/part1/step6.png}
    \end{center}

    \begin{stopbox}
        Before proceeeding to the next step, confirm that \textbf{ubuntu} appears underneath the \textit{Allocated}
        section.
    \end{stopbox}

    \item In the \textit{Flavor} tab, click the $\uparrow$ symbol in the same row as \textbf{m1.small}. Click
    \textbf{Next}.

    \begin{center}
        \includegraphics[width=\linewidth]{images/part1/step7.png}
    \end{center}

    \begin{stopbox}
        Before proceeding to the next step, confirm that \textbf{m1.small} appears underneath the \textit{Allocated}
        section.
    \end{stopbox}

    \item In the \textit{Networks} tab, click the $\uparrow$ symbol in the same row as \textbf{shared}. Click
    \textbf{Launch Instance}.

    \begin{center}
        \includegraphics[width=\linewidth]{images/part1/step8.png}
    \end{center}

    \begin{stopbox}
        Before proceeding to the next step, confirm that \textbf{shared} appeears underneath the \textit{Allocated}
        section.
    \end{stopbox}

    \item Access the instance's console by clicking on \textbf{instance1} under the \textit{Instance Name} column. Then,
    navigate to the \textit{Console} tab if you are not directed there automatically. Click on \textbf{Click here to
    show only the console}.

    \begin{center}
        \includegraphics[width=\linewidth]{images/part1/step9.png}
    \end{center}

    \item Log into the instance as \textbf{root} with the password \textbf{secret}.

    \begin{center}
        \includegraphics[width=\linewidth]{images/part1/step10.png}
    \end{center}

    \begin{notebox}
        It may take several minutes for the instance to fully boot up and present a login prompt.
    \end{notebox}

    \item Now, we will make a change to the instance and create a snapshot. Create the \textbf{/root/hello.txt} file
    with the contents \textbf{Hello, world!}.
    \begin{lstlisting}
        root@instance1:~# echo 'Hello, world!' > /root/hello.txt
    \end{lstlisting}

    \begin{center}
        \includegraphics[width=\linewidth]{images/part1/step11.png}
    \end{center}

    \item Now, navigate back to \textbf{Project $>$ Compute $>$ Instances}. Before creating a snapshot, click the
    dropdown next to \textbf{Create Snapshot} in the same row as \textbf{instance1}, then click \textbf{Shut Off
    Instance}. This will prevent any inconsistencies in the resulting snapshot.

    \begin{center}
        \includegraphics[width=\linewidth]{images/part1/step12.png}
    \end{center}

    \begin{tipbox}
        A ``live snapshot'' is a snapshot of a running instance, which may only include a snapshot of the disk, while
        some OS state may be lost.
    \end{tipbox}
    \begin{notebox}
        Stopping instances and otherwise changing their running states will be explored further later in this lab.
    \end{notebox}

    \item When the \textit{Status} column shows that \textbf{instance1} is \textbf{Shutoff}, click \textbf{Create
    Snapshot}.

    \begin{center}
        \includegraphics[width=\linewidth]{images/part1/step13.png}
    \end{center}

    \item In the \textbf{Create Snapshot} dialog, enter \textbf{instance1-snapshot} in the \textit{Snapshot Name} field.
    Click \textbf{Create Snapshot}.

    \begin{center}
        \includegraphics[width=\linewidth]{images/part1/step14.png}
    \end{center}

    \begin{stopbox}
        When you create the snapshot, you will be redirected to \textbf{Projects $>$ Compute $>$ Images}. Wait until
        \textbf{instance1-snapshot} is \textbf{Active} before proceeding.
    \end{stopbox}

    \item Navgiate back to the \textbf{Instances} page. The instance is no longer needed, so select the checkbox next to
    \textbf{instance1} and click \textbf{Delete Instances}.

    \begin{center}
        \includegraphics[width=\linewidth]{images/part1/step15.png}
    \end{center}

    \item In the \textbf{Confirm Delete Instances} dialog, click \textbf{Delete Instances}.

    \begin{center}
        \includegraphics[width=\linewidth]{images/part1/step16.png}
    \end{center}

    \item To verify that the snapshot works correctly, launch another instance named \textbf{instance1} using the
    snapshot. Follow the same steps that were used to create \textbf{instance1}. However, in the \textit{Source} tab of
    the \textbf{Launch Instance} dialog, select \textbf{Image Snapshot} under \textit{Select Boot Source} click the
    $\uparrow$ symbol next to \textbf{instance1-snapshot}.

    \begin{center}
        \includegraphics[width=\linewidth]{images/part1/step17.png}
    \end{center}

    \begin{tipbox}
        The snapshot will also appear on the \textbf{Project $>$ Compute $>$ Images} page. It should say
        \textbf{Snapshot} in the \textit{Type} column. For an alternative method of launching an image using the
        snapshot, navigate to this page, click \textbf{Launch} in the same row as the snapshot, and enter the required
        information in the following dialog. The snapshot can also be deleted from here.
    \end{tipbox}

    \item Open the instance's console and log in with the username \textbf{root} and password \textbf{secret}. Check
    that the file created in the previous instance also exists on this instance.
    \begin{lstlisting}
        root@instance1:~# cat /root/hello.txt
    \end{lstlisting}

    \begin{center}
        \includegraphics[width=\linewidth]{images/part1/step18.png}
    \end{center}

    \begin{notebox}
        It may take several minutes for the instance to fully boot up and present a login prompt.
    \end{notebox}

    \item Exit the instance's console, navigate back to \textbf{Project $>$ Compute $>$ Instances}, and delete the
    instance.

    \begin{center}
        \includegraphics[width=\linewidth]{images/part1/step19.png}
    \end{center}

    \item You will create another snapshot with the \textit{OpenStack Unified CLI} in the next task, so
    \textbf{instance1-snapshot} can safely be deleted. Navigate to \textbf{Project $>$ Compute $>$ Images}, select
    \textbf{instance1-snapshot}, and click \textbf{Delete Images}.

    \begin{center}
        \includegraphics[width=\linewidth]{images/part1/step20.png}
    \end{center}

    \item In the \textbf{Confirm Delete Image} dialog, click \textbf{Delete Image}.
    
    \begin{center}
        \includegraphics[width=\linewidth]{images/part1/step21.png}
    \end{center}

    \item Log out of the dashboard, close the browser window, and continue to the next task.
\end{enumerate}

%%%%%%%%%%%
% Section 2
%%%%%%%%%%%
\section{Creating a Snapshot Using the OpenStack Unififed CLI}
\label{sec:creating_a_snapshot_cli}
In this task, you will repeat the steps from the previous task in the \textit{OpenStack Unified CLI}.

\begin{enumerate}
    \item Open a terminal window and source the keystone credentials for the \textbf{admin} user.
    \begin{lstlisting}
        ubuntu@workstation:~$ source ~/keystonerc-admin
    \end{lstlisting}

    \begin{center}
        \includegraphics[width=\linewidth]{images/part2/step1.png}
    \end{center}

    \item List the current instances. The list should be empty.
    \begin{lstlisting}
        [ubuntu@workstation (keystone-admin)]:~$ openstack server list
    \end{lstlisting}

    \begin{center}
        \includegraphics[width=\linewidth]{images/part2/step2.png}
    \end{center}

    \item Now, we will create the same snapshot as before from the command line. Launch an instance. Use the
    \textbf{ubuntu} image, the \textbf{m1.small} flavor, and the \textbf{shared} network.
    \begin{lstlisting}
        [ubuntu@workstation (keystone-admin)]:~$ openstack server create \
        > --image ubuntu \
        > --flavor m1.small \
        > --nic net-id=shared \
        instance2
    \end{lstlisting}

    \begin{center}
        \includegraphics[width=\linewidth]{images/part2/step3.png}
    \end{center}

    \item List the instances again to ensure it was created correctly.
    \begin{lstlisting}
        [ubuntu@workstation (keystone-admin)]:~$ openstack server list \
        > --max-width 80
    \end{lstlisting}

    \begin{center}
        \includegraphics[width=\linewidth]{images/part2/step4.png}
    \end{center}

    \begin{tipbox}
        When typing the command, make sure there is a space between \textbf{list} and the \textbf{\textbackslash}
        character, and press \textbf{Enter} to get the \textbf{$>$} and continue typing the rest of the command.
    \end{tipbox}

    \item Show the URL to the console of the instance. Right-click the URL and click \textbf{Open Link}.
    \begin{lstlisting}
        [ubuntu@workstation (keystone-admin)]:~$ openstack console url show instance2
    \end{lstlisting}

    \begin{center}
        \includegraphics[width=\linewidth]{images/part2/step5.png}
    \end{center}

    \item Log in to \textbf{instance2} as \textbf{root} with the password \textbf{secret}.
    
    \begin{center}
        \includegraphics[width=\linewidth]{images/part2/step6.png}
    \end{center}

    \begin{notebox}
        It may take several minutes for the instance to fully boot up and present a login prompt.
    \end{notebox}

    \item Create the \textbf{/root/hello.txt} file with the contents \textbf{Hello, world!}.
    \begin{lstlisting}
        root@instance2:~# echo 'Hello, world!' > /root/hello.txt
    \end{lstlisting}

    \begin{center}
        \includegraphics[width=\linewidth]{images/part2/step7.png}
    \end{center}

    \item Close the browser window and return focus to the terminal window. Stop the instance before making a snapshot.
    \begin{lstlisting}
        [ubuntu@workstation (keystone-admin)]:~$ openstack server stop instance2
    \end{lstlisting}

    \begin{center}
        \includegraphics[width=\linewidth]{images/part2/step8.png}
    \end{center}

    \item List the current images. The list should have two items.
    \begin{lstlisting}
        [ubuntu@workstation (keystone-admin)]:~$ openstack image list
    \end{lstlisting}

    \begin{center}
        \includegraphics[width=\linewidth]{images/part2/step9.png}
    \end{center}

    \item Make a snapshot of the instance.
    \begin{lstlisting}
        [ubuntu@workstation (keystone-admin)]:~$ openstack server image create \
        > instance2 \
        > --name instance2-snapshot \
        > --max-width 80
    \end{lstlisting}

    \begin{center}
        \includegraphics[width=\linewidth]{images/part2/step10.png}
    \end{center}

    \item List the current images again to ensure the snapshot was create properly.
    \begin{lstlisting}
        [ubuntu@workstation (keystone-admin)]:~$ openstack image list
    \end{lstlisting}

    \begin{center}
        \includegraphics[width=\linewidth]{images/part2/step11.png}
    \end{center}

    \item To verify the correctness of the snapshot, first delete the instance.
    \begin{lstlisting}
        [ubuntu@workstation (keystone-admin)]:~$ openstack server delete instance2
    \end{lstlisting}

    \begin{center}
        \includegraphics[width=\linewidth]{images/part2/step12.png}
    \end{center}

    \item Confirm the deletion of the instance.
    \begin{lstlisting}
        [ubuntu@workstation (keystone-admin)]:~$ openstack server list
    \end{lstlisting}

    \begin{center}
        \includegraphics[width=\linewidth]{images/part2/step13.png}
    \end{center}

    \item Now, recreate the instance, using \textbf{instance2-snapshot} as the image.
    \begin{lstlisting}
        [ubuntu@workstation (keystone-admin)]:~$ openstack server create \
        > --image instance2-snapshot \
        > --flavor m1.small \
        > --nic net-id=shared \
        instance2
    \end{lstlisting}

    \begin{center}
        \includegraphics[width=\linewidth]{images/part2/step14.png}
    \end{center}

    \item Using the same steps as before, log in to the instance's console as \textbf{root} using the password
    \textbf{secret}. Verify that the \textbf{/root/hello.txt} file exists.
    \begin{lstlisting}
        root@instance2:~# cat /root/hello.txt
    \end{lstlisting}

    \begin{center}
        \includegraphics[width=\linewidth]{images/part2/step15.png}
    \end{center}

    \begin{notebox}
        It may take several minutes for the instance to fully boot up and present a login prompt.
    \end{notebox}

    \item This instance will be used in the following sections. Leave both the browser and terminal windows open and
    continue to the next task.
    
\end{enumerate}

% TODO: In the case that volumes are added to these labs, rescuing instances should be discussed in the following two
% sections.

%%%%%%%%%%%
% Section 3
%%%%%%%%%%%
\section{Managing the Running State of an Instance Using the Horizon Dashboard}
\label{sec:managing_the_running_state_of_an_instance_web}
OpenStack allows for managing the running and power state of instances in several ways, and each method may be useful in
different situations. In this task, you will manage the running and power state of an instance by starting, stopping,
pausing, suspending, resuming, shelving, unshelving, and rebooting the instance using the \textit{Horizon Dashboard}.

\begin{enumerate}
    \item Open a new tab in the browser window and navigate to \textbf{192.168.1.20}. Log in as \textbf{admin} with the
    password \textbf{secret}.

    \item Pausing an instance is one way to manage the running state of an OpenStack instance. When an instance is
    paused, its operation is frozen, and its state and memory are preserved in the RAM of the underlying compute node.
    Pausing an instance does not release its resources. When the instance is resumed, it will pick up any processes
    where they left off. To view the effects of pausing an instance, first focus back on the tab with the instance's
    console and continuously ping the DHCP server.
    \begin{lstlisting}
        root@instance2:~# ping 192.168.233.2
    \end{lstlisting} 
    
    \begin{center}
        \includegraphics[width=\linewidth]{images/part3/step2.png}
    \end{center}

    \begin{tipbox}
        Pausing an instance is useful when the operation of an instance needs to be interrupted but its state should be
        kept intact. For example, an instance might be paused while making changes to the underlying infrastructure
        to prevent disrupting processes and requiring applications or the instance to be restarted. Pausing an instance
        is similar to putting a computer in sleep mode.
    \end{tipbox}

    \item To pause the instance, navigate to \textbf{Project $>$ Compute $>$ Instances}, click the dropdown next to
    \textbf{Create Snapshot} in the same row as \textbf{instance2}, and click \textbf{Pause Instance}.

    \begin{center}
        \includegraphics[width=\linewidth]{images/part3/step3.png}
    \end{center}

    \begin{notebox}
        You may have to scroll down to find the option.
    \end{notebox}

    \item Now, view the console again to see that it is frozen and no more ping replies are appearing.
    
    \begin{center}
        \includegraphics[width=\linewidth]{images/part3/step4.png}
    \end{center}

    \item Navigate back to the \textbf{Instances} page. Click the dropdown next to \textbf{Create Snapshot} in the same
    row as \textbf{instance2} and click \textbf{Resume Instance}.

    \begin{center}
        \includegraphics[width=\linewidth]{images/part3/step5.png}
    \end{center}

    \item View the console again to see that the ping replies have resumed. Press \textbf{Ctrl+C} to stop the
    \textbf{ping} command.

    \begin{center}
        \includegraphics[width=\linewidth]{images/part3/step6.png}
    \end{center}

    \item Suspending an instance is similar to pausing an instance. The main difference is that the instance's state is
    written to a persistent disk of the underlying compute node rather than memory. This means the state can be
    preserved even if the compute node loses power during the suspension. Suspending an instance does not release its
    resources. When the instance is resumed, it will pick up any processes where they left off. To view the effects of
    suspending an instance, perform the same experiment as before. Focus on the tab with the instance's console and
    continuously ping the DHCP server.
    \begin{lstlisting}
        root@instance2:~# ping 192.168.233.2
    \end{lstlisting} 

    \begin{center}
        \includegraphics[width=\linewidth]{images/part3/step7.png}
    \end{center}

    \begin{tipbox}
        Suspending an instance is useful in similar situations as pausing. However, suspending an image allows the
        compute node to be rebooted or migrated without disrupting the processes of the instance and requiring
        applications or the instance to be restarted. Suspending an instance is similar to putting a computer in
        hibernation mode.
    \end{tipbox}

    \item To suspend the instance, navigate back to \textbf{Project $>$ Compute $>$ Instances}, click the dropdown next
    to \textbf{Create Snapshot}, and click \textbf{Suspsend Instance}.

    \begin{center}
        \includegraphics[width=\linewidth]{images/part3/step8.png}
    \end{center}

    \begin{notebox}
        You may have to scroll down to find the option.
    \end{notebox}

    \item View the console again to see that the connection has been ended. When the instance is resumed, a new
    connection will be created, and this tab will still be unresponsive. Close the tab containing the instance console.

    \begin{center}
        \includegraphics[width=\linewidth]{images/part3/step9.png}
    \end{center}

    \item Navigate back to the \textbf{Instances} page. Click the dropdown next to \textbf{Create Snapshot} in the same
    row as \textbf{instance2} and click \textbf{Resume Instance}.

    \begin{center}
        \includegraphics[width=\linewidth]{images/part3/step10.png}
    \end{center}

    \item Click on \textbf{instance2} and select the \textbf{Console} tab to see that the ping replies have resumed.
    Press \textbf{Ctrl+C} to stop the \textbf{ping} process.

    \begin{center}
        \includegraphics[width=\linewidth]{images/part3/step11.png}
    \end{center}

    \item Shutting off or stopping an instance turns off the instance. The instance state and any data stored in the
    instance's RAM will be lost. Stopping an instance does not release its resources. To stop an instance, navigate to
    the \textbf{Instances} page. Click the dropdown next to \textbf{Create Snapshot} in the same row as
    \textbf{instance2} and click \textbf{Shut Off Instance}.

    \begin{center}
        \includegraphics[width=\linewidth]{images/part3/step12.png}
    \end{center}

    \begin{notebox}
        You may have to scroll down to find the option.
    \end{notebox}
    
    \item In the \textbf{Confirm Shutt Off Instance} dialog, click \textbf{Shut Off Instance}.

    \begin{center}
        \includegraphics[width=\linewidth]{images/part3/step13.png}
    \end{center}

    \item When the power state of the instance indicates that it is shut off, the \textbf{Create Snapshot} button will
    become \textbf{Start Instance}. Click this button to turn the instance back on.

    \begin{center}
        \includegraphics[width=\linewidth]{images/part3/step14.png}
    \end{center}

    \begin{tipbox}
        In addition to shutting off an instance, an instance can also be soft or hard rebooted, or turned off and back
        on. A soft reboot allows the instance to perform a graceful shutdown, while hard rebooting an instance is
        analogous to pulling the power cord from a computer.
    \end{tipbox}

    \item Close the browser window and continue to the next task.
\end{enumerate}

%%%%%%%%%%%
% Section 4
%%%%%%%%%%%
\section{Managing the Running State of an Instance Using the OpenStack Unififed CLI}
\label{sec:managing_the_power_state_of_an_instance_cli}
In this task, you will repeat the steps from the previous task in the \textit{OpenStack Unified CLI}.

\begin{enumerate}
    \item If a terminal window is not already open, open one and source the keystone credentials for the \textbf{admin}
    user.
    \begin{lstlisting}
        ubuntu@workstation:~$ source ~/keystonerc-admin
    \end{lstlisting}

    \begin{center}
        \includegraphics[width=\linewidth]{images/part4/step1.png}
    \end{center}

    \item Show the URL to the console of the instance. Right-click the URL and click \textbf{Open Link}.
    \begin{lstlisting}
        [ubuntu@workstation (keystone-admin)]:~$ openstack console url show instance2
    \end{lstlisting}

    \begin{center}
        \includegraphics[width=\linewidth]{images/part4/step2.png}
    \end{center}

    \item Log in to \textbf{instance2} as \textbf{root} with the password \textbf{secret}.
    
    \begin{center}
        \includegraphics[width=\linewidth]{images/part4/step3.png}
    \end{center}

    \item Continuously ping the DHCP server to see the effects of pausing an instance.
    \begin{lstlisting}
        root@instance2:~# ping 192.168.233.2
    \end{lstlisting}

    \begin{center}
        \includegraphics[width=\linewidth]{images/part4/step4.png}
    \end{center}

    \item Focus back on the terminal window and pause the instance.
    \begin{lstlisting}
        [ubuntu@workstation (keystone-admin)]:~$ openstack server pause instance2
    \end{lstlisting}

    \begin{center}
        \includegraphics[width=\linewidth]{images/part4/step5.png}
    \end{center}

    \item Now, view the browser window again to see that the instance is frozen and no more ping replies are appearing.

    \begin{center}
        \includegraphics[width=\linewidth]{images/part4/step6.png}
    \end{center}

    \item Focus back on the terminal window and resume the instance
    \begin{lstlisting}
        [ubuntu@workstation (keystone-admin)]:~$ openstack server unpause instance2
    \end{lstlisting}

    \begin{center}
        \includegraphics[width=\linewidth]{images/part4/step7.png}
    \end{center}

    \item View the browser window again to see that more ping replies are coming in. Press \textbf{Ctrl+C} to stop the
    \textbf{ping} process.

    \begin{center}
        \includegraphics[width=\linewidth]{images/part4/step8.png}
    \end{center}

    \item Start the \textbf{ping} process again to view the effects of suspending an instance.
    \begin{lstlisting}
        root@instance2:~# ping 192.168.233.2
    \end{lstlisting}

    \begin{center}
        \includegraphics[width=\linewidth]{images/part4/step9.png}
    \end{center}

    \item Focus back on the terminal window and suspend the instance.
    \begin{lstlisting}
        [ubuntu@workstation (keystone-admin)]:~$ openstack server suspend instance2
    \end{lstlisting}

    \begin{center}
        \includegraphics[width=\linewidth]{images/part4/step10.png}
    \end{center}

    \item View the console again to see that the connection has been ended. When the instance is resumed, a new
    connection will be created, and this tab will still be unresponsive. Close the browser window.

    \begin{center}
        \includegraphics[width=\linewidth]{images/part4/step11.png}
    \end{center}

    \item Focus back on the terminal window and resume the instance.
    \begin{lstlisting}
        [ubuntu@workstation (keystone-admin)]:~$ openstack server resume instance2
    \end{lstlisting}

    \begin{center}
        \includegraphics[width=\linewidth]{images/part4/step12.png}
    \end{center}

    \item Show the URL to the console of the instance. Right-click the URL and click \textbf{Open Link}.
    \begin{lstlisting}
        [ubuntu@workstation (keystone-admin)]:~$ openstack console url show instance2
    \end{lstlisting}

    \begin{center}
        \includegraphics[width=\linewidth]{images/part4/step13.png}
    \end{center}

    \item Log in to \textbf{instance2} as \textbf{root} with the password \textbf{secret}.
    
    \begin{center}
        \includegraphics[width=\linewidth]{images/part4/step14.png}
    \end{center}

    \item Note that more ping replies are now coming in. Press \textbf{Ctrl+C} to stop the \textbf{instance}, and
    close the browser window.

    \begin{center}
        \includegraphics[width=\linewidth]{images/part4/step15.png}
    \end{center}

    \item Focus back on the terminal window and confirm that \textbf{instance2} is listed as \textbf{ACTIVE}.
    \begin{lstlisting}
        [ubuntu@workstation (keystone-admin)]:~$ openstack server list \
        > --max-width 80
    \end{lstlisting}
    
    \begin{center}
        \includegraphics[width=\linewidth]{images/part4/step16.png}
    \end{center}

    \item Stop the instance.
    \begin{lstlisting}
        [ubuntu@workstation (keystone-admin)]:~$ openstack server stop instance2
    \end{lstlisting}

    \begin{center}
        \includegraphics[width=\linewidth]{images/part4/step17.png}
    \end{center}

    \item Verify that the status of \textbf{instance2} is \textbf{SHUTOFF}.
    \begin{lstlisting}
        [ubuntu@workstation (keystone-admin)]:~$ openstack server list \
        > --max-width 80
    \end{lstlisting}

    \begin{center}
        \includegraphics[width=\linewidth]{images/part4/step18.png}
    \end{center}

    \item Power the instance back on.
    \begin{lstlisting}
        [ubuntu@workstation (keystone-admin)]:~$ openstack server start instance2
    \end{lstlisting}

    \begin{center}
        \includegraphics[width=\linewidth]{images/part4/step19.png}
    \end{center}

    \item Verify that the status of \textbf{instance2} is \textbf{ACTIVE}.
    \begin{lstlisting}
        [ubuntu@workstation (keystone-admin)]:~$ openstack server list \
        > --max-width 80
    \end{lstlisting}

    \begin{center}
        \includegraphics[width=\linewidth]{images/part4/step20.png}
    \end{center}

    \item The lab is now complete.

\end{enumerate}
\end{document}
